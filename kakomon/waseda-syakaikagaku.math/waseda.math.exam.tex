\documentclass[10pt,fleqn,a4paper]{jsarticle}
\usepackage[dvipdfmx]{graphicx,color}
\usepackage{ascmac,amsmath,amssymb,amstext}
\usepackage{tikz,multicol,float,tikz-3dplot}
\usetikzlibrary{positioning,intersections,calc,arrows.meta,math,angles}
\usepackage{zogeny,ceo,comment}
\newcommand{\ans}[1]{\mbox{\boldmath{$#1$}}}
\renewcommand{\baselinestretch}{1.3} 
\tdplotsetmaincoords{60}{110}%{xy平面からどれだけ上にいるか(90度から-90度)}{z軸中心にどれだけ左右に回転するか}
\title{早稲田大学ー社会科学部ー過去問演習と良問演習}
\author{大島 遙斗}
\begin{document}
\maketitle
\newpage
{\LARGE はじめに}\\
長い長い受験勉強もとうとう終わりが見えてきました.今これを書いているのは12月の14日の深夜です.今から,2ヶ月後の理瀬さんは
もう大学入試を終えています.さて,どんな景色が見えているのでしょうか.\\\\
今から志望校の合格にできることは,たくさん問題をといてパターンをたくさん身につけるのではなく,
\begin{center}
\ans{出会った問題から何を学ぶか?}
\end{center}です.\\
1日1日を大切に頑張っていきましょう.\\\\
また,勉強の相談やメンタル的にしんどいなどあれば気軽に電話なり,直接僕に話しかけたりして教えてください!いつでも相談にのります!最終局面に向けて
一緒に頑張りましょう.\\
\hspace{14cm}大島 遙斗
\newpage
\begin{center}
    {\LARGE 本書の使い方}
\end{center}
まず,このテキストは\ans{早稲田の三年分の過去問と僕が選んだ良問}の二部から構成されています.\\
それぞれ1週間のうちに解いてきてください.\\
・過去問について\\
 言わずもがなよく復習してください.何がダメで次はどうすれば解けるようになるのか,どのような発想があったら解くことができていたのか,
を考えながら復習しましょう.\\
・良問について\\
 立教が一応のところの第一志望だと思うので掲載する問題は,\hyoujyun 〜\yayanan の問題を集めます.また,以下の記号を用います.\\
A$\to 基本的な問題.教科書レベル.\\
B\to 標準的な問題.ぜひ解き切って欲しい問題.\\
C\to $難問.この問題が解けなくても,周りの受験生とはあまり差がつかない.\\
 例えば,次のように問題の横に書いたら次のように捉えてください.\\
\kagil B15\kagir$=$標準的な問題で目標解答時間は,15分\\
また,本題の他に\fbox{\kurosankakub 類題演習\kurosankakua}というものを掲載します.必ずしも解いてくる必要はありません.自分の実力向上に役立ててください.
\newpage
\tableofcontents
\newpage
\section{早稲田大学過去問題編}
\subsection{2023年実施}
\begin{center}
    {\LARGE 2023年度:数学}
\end{center}
\begin{center}
{\LARGE 注意事項}\\
\end{center}
1. \\
2. \\
3. \\
4. \\
  \kakkoichi\\
  \kakkoni\\
  \kakkosan\\
  \kakkoshi\\
  \kakkogo\\
  \kakkoroku\\
5. \\
6. \\
7. \\
8. \\
9. \\
\newpage
\begin{center}
    計算用紙
\end{center}
\newpage
\begin{center}
    計算用紙
\end{center}
\newpage
\begin{center}
    計算用紙
\end{center}
\newpage
\begin{center}
    計算用紙
\end{center}
\newpage
\noindent
\reibanichi\\
 曲線$y=ax^2+b上にx座標がpである点Pをとり,点Pにおける接線を\ell とする.$\\
ただし,定数$a,bはa>0,b>0を満たすとする.次の問に答えよ.$\\\\
\kakkoichi 接線$\ell の方程式をa,b,pを用いて表せ.\\
\kakkoni  接線\ell とy=ax^2で囲まれた部分の面積Sをa,bを用いて表せ.\\
\kakkosan 接線\ell と曲線y=ax^2+\dfrac{b}{2}で囲まれた図形の面積をS^\prime としたとき,S^\prime をSを用いて表せ.\\
\kakkoshi 接線\ell と曲線y=ax^2+cで囲まれた部分の面積S^{\prime\prime} とする.S^{\prime\prime}=\dfrac{S}{2}のとき,
cをa,bを用いて表せ.ただし,b>cとする.$\\\\\\\\\\\\\\\\\\\\
\reibanni\\
 定数$mに対してx,y,zの方程式$
\begin{center}
    $xyz+x+y+z=xy+yz+zx+m\hspace{2cm}\cdots\maruichi$
\end{center}
を考える.次の問に答えよ.\\\\
\kakkoichi $m=1のとき\maruichi 式をみたす実数x,y,zの組をすべて求めよ.$\\
\kakkoni $m=5のとき\maruichi 式を満たす実数x,y,zの組をすべて求めよ.ただし,x\leq y\leq zとする.$\\
\kakkosan $xyz=x+y+zを満たす整数x,y,zの組をすべて求めよ.ただし,0<x\leq y\leq zとする.$
\newpage
\reibansan\\
 $a=\sqrt[3]{5\sqrt{2}+7}-\sqrt[3]{5\sqrt{2}-7}とする.次の問に答えよ.$\\\\
\kakkoichi $a^3をaの1次式で表せ.$\\
\kakkoni $aは整数であることを示せ.$\\
\kakkosan $b=\sqrt[3]{5\sqrt{2}+7}+\sqrt[3]{5\sqrt{2}-7}とするとき,bを超えない最大の整数を求めよ.$\\\\\\\\
\begin{center}
    {\LARGE\kagil 以 下 余 白\kagir}
\end{center}
\newpage
\section{良問集問題編}
\subsection{第1回}
まずは,単純な計算問題で肩慣らしといきましょう.おっとその前に公式の確認です.\\
\koushikia\\
\doichi  $\ans{|X|=k\doti X=\pm kかつk\geq0}\\
\doni \ans{|X|<k\doti -k<X<k}\\
\dosan \ans{|X|>k\doti X>kまたはX<-k}$\\
\begin{itembox}
[l]{\noindent\ba{1}.\kagil 絶対値と不等式(A10)\kagir}
以下の不等式をそれぞれ解け.\\
\kakkoichi $|x+3|\geq|x-2|$\hspace{7cm}(25 宮崎大・教,農)\\%bunkei-49
\kakkoni $|4x-1|<|x+3|$\hspace{7cm}(25 福島大)%bunkei-50
\end{itembox}\\\\\\\\\\\\\\\\\\\\\\

\begin{itembox}
    [l]{\fbox{\kurosankakub 類題演習\kurosankakua}}
\kakkoichi 不等式$|2x-3|\leq2の解を求めよ.さらに,不等式|2x-3|\leq2\leq\dfrac{1-3a}{3}x-1の解が1\leq x\leq\dfrac{5}{2}となるような
定数aの値の範囲を求めよ.$\hspace{2cm}(25 同志社女子大)\\
\kakkoni $|x|+|x-3|<4を解け.\hspace{6cm}(25 大東文化大)$
\end{itembox}
\newpage
\begin{itembox}
    [l]{\ba{2}}
    あいうえお
\end{itembox}



\end{document}