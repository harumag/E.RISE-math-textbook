\documentclass[10pt,fleqn,a4paper]{jsarticle}
\usepackage[dvipdfmx]{graphicx,color}
\usepackage{ascmac,amsmath,amssymb,amstext}
\usepackage{tikz,multicol,float,tikz-3dplot}
\usetikzlibrary{positioning,intersections,calc,arrows.meta,math,angles}
\usepackage{zogeny,ceo,comment}
\newcommand{\ans}[1]{\mbox{\boldmath{$#1$}}}
\renewcommand{\baselinestretch}{1.3} 
\tdplotsetmaincoords{60}{110}%{xy平面からどれだけ上にいるか(90度から-90度)}{z軸中心にどれだけ左右に回転するか}
\title{早稲田大学ー社会科学部ー過去問演習と良問演習}
\author{大島 遙斗}
\begin{document}
\maketitle
\newpage
{\LARGE はじめに}\\
長い長い受験勉強もとうとう終わりが見えてきました.今これを書いているのは12月の14日の深夜です.今から,2ヶ月後の理瀬さんは
もう大学入試を終えています.さて,どんな景色が見えているのでしょうか.\\\\
今から志望校の合格にできることは,たくさん問題をといてパターンをたくさん身につけるのではなく,
\begin{center}
\ans{出会った問題から何を学ぶか?}
\end{center}です.\\
1日1日を大切に頑張っていきましょう.\\\\
また,勉強の相談やメンタル的にしんどいなどあれば気軽に電話なり,直接僕に話しかけたりして教えてください!いつでも相談にのります!最終局面に向けて
一緒に頑張りましょう.\\
\hspace{14cm}大島 遙斗
\newpage
\tableofcontents
\newpage
\section{早稲田大学過去問題編}
\subsection{2023年実施}
\begin{center}
    {\LARGE 2023年度:数学}
\end{center}
\begin{center}
{\LARGE 注意事項}\\
\end{center}
1. \\
2. \\
3. \\
4. \\
  \kakkoichi\\
  \kakkoni\\
  \kakkosan\\
  \kakkoshi\\
  \kakkogo\\
  \kakkoroku\\
5. \\
6. \\
7. \\
8. \\
9. \\
\newpage
\begin{center}
    計算用紙
\end{center}
\newpage
\begin{center}
    計算用紙
\end{center}
\newpage
\begin{center}
    計算用紙
\end{center}
\newpage
\begin{center}
    計算用紙
\end{center}
\newpage
\noindent
\reibanichi\\
 曲線$y=ax^2+b上にx座標がpである点Pをとり,点Pにおける接線を\ell とする.$\\
ただし,定数$a,bはa>0,b>0を満たすとする.次の問に答えよ.$\\\\
\kakkoichi 接線$\ell の方程式をa,b,pを用いて表せ.\\
\kakkoni  接線\ell とy=ax^2で囲まれた部分の面積Sをa,bを用いて表せ.\\
\kakkosan 接線\ell と曲線y=ax^2+\dfrac{b}{2}で囲まれた図形の面積をS^\prime としたとき,S^\prime をSを用いて表せ.\\
\kakkoshi 接線\ell と曲線y=ax^2+cで囲まれた部分の面積S^{\prime\prime} とする.S^{\prime\prime}=\dfrac{S}{2}のとき,
cをa,bを用いて表せ.ただし,b>cとする.$\\\\\\\\\\\\\\\\\\\\
\reibanni\\
 定数$mに対してx,y,zの方程式$
\begin{center}
    $xyz+x+y+z=xy+yz+zx+m\hspace{2cm}\cdots\maruichi$
\end{center}
を考える.次の問に答えよ.\\\\
\kakkoichi $m=1のとき\maruichi 式をみたす実数x,y,zの組をすべて求めよ.$\\
\kakkoni $m=5のとき\maruichi 式を満たす実数x,y,zの組をすべて求めよ.ただし,x\leq y\leq zとする.$\\
\kakkosan $xyz=x+y+zを満たす整数x,y,zの組をすべて求めよ.ただし,0<x\leq y\leq zとする.$
\newpage
\reibansan\\
 $a=\sqrt[3]{5\sqrt{2}+7}-\sqrt[3]{5\sqrt{2}-7}とする.次の問に答えよ.$\\\\
\kakkoichi $a^3をaの1次式で表せ.$\\
\kakkoni $aは整数であることを示せ.$\\
\kakkosan $b=\sqrt[3]{5\sqrt{2}+7}+\sqrt[3]{5\sqrt{2}-7}とするとき,bを超えない最大の整数を求めよ.$\\\\\\\\
\begin{center}
    {\LARGE\kagil 以 下 余 白\kagir}
\end{center}








\end{document}