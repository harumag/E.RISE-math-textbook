\documentclass[10pt,fleqn,a4paper]{jsarticle}
\usepackage[dvipdfmx]{graphicx,color}
\usepackage{ascmac,amsmath,amssymb,amstext}
\usepackage{tikz,multicol,float,tikz-3dplot}
\usetikzlibrary{positioning,intersections,calc,arrows.meta,math,angles}
\usepackage{zogeny,ceo,comment}
\newcommand{\ans}[1]{\mbox{\boldmath{$#1$}}}
\renewcommand{\baselinestretch}{1.3} 
\tdplotsetmaincoords{60}{110}%{xy平面からどれだけ上にいるか(90度から-90度)}{z軸中心にどれだけ左右に回転するか}
\title{立教大-社会科学部-過去問演習}
\author{大島}
\begin{document}
\maketitle
\newpage
{\LARGE 過去問演習の意味とは}\\
 受験生が受験勉強をする中で誰しも通るものが,\ans{過去問演習}である.\\
ただし,その過去問演習は,\ans{回数をこなすだけ}の勉強になってはいないだろうか.そこで,私のおすすめの過去問への取り組み方を少し
述べようと思う.\\\\
過去問演習とは,次の目的を達成するために用いるべし.\\

    {\large \noindent\ans{1}.受験校の難易度・傾向を掴む.}\\
    {\large \ans{2}.受験校の教授は,どのような\ans{数学的な発想・考え方をすることを要求しているのか?}を把握する.
    (傾向に合わせた思考をできるようになる.)}\\
    {\large \ans{3}.時間配分を考える.}\\
    {\large \ans{4}.どの程度できれば御の字なのかを把握し,その正答率を目指して解く.}\\
    {\large \ans{5}.自分で採点せず,他の人に添削をお願いする.(自分では,良いように思えても実際,ダメなことがよくある.)}\\
    {\large\ans{6}.過去問演習は,2週目まで.(周回して,受かる気になっているだけ.過去問で出題された問題はほぼ出ない.)}\\
    {\large \ans{7}.良い点が取れなくても落ち込まない.(相性がある.その結果に一喜一憂している暇はない.その時間を勉強に充てよ.)}\\
    {\large \ans{8}.最後まで,自分を信じて取り組むこと.(今,合格圏内にいる者はほとんどいない.(僕も,そうだった)最後まで,諦めない!!)}\\\\

    上にあ挙げたようなことを意識して過去問題演習に取り組むのが良いだろう.\\
    最終局面が迫っています.頑張りましょう.\\
    大島 遙斗
    \newpage
    \tableofcontents
    \newpage
    余白
    \newpage
    \begin{center}
        2021年度
    \end{center}
    \begin{center}
        

    {\LARGE \ans{A 数 学 問 題}}
    \end{center}
    \begin{center}
        \ans{注 意}
        \end{center}
        \begin{center}
            
        
        
            1.試験開始の指示があるまでこの問題冊子を開いてはいけません。\\
            2.解答用紙はすべて\ans{黒鉛筆または黒のシャープペンシル}で記入することになっています。黒鉛筆・消しゴムを忘れた人は
            監督に申し出てください。(万年筆・ボールペン・サインペンなどを使用してはいけません。)\\
            3.この\\
            4.解答用紙\\
            5.解答は,\\
            6.解答用紙を\\
            7.計算には,\\
            8.この問題冊子
\end{center}

        \newpage
    \section{2021年度}
    い
    \newpage









    \section{2022年度}
    ろ
    \newpage






    \section{2023年度}
    は
    \newpage




    \section{2024年度}
制限時間:60分 解答用紙:A3一枚\\
{\LARGE \tokeichi.} 下記の空欄ア〜コにあてはまる数または式を解答用紙(省略)の所定欄に記入せよ.\\
$\tokeiichi 1\leq x\leq8の範囲において,関数y=\p{\log_2x}^2-8\log_2x-20はx=\Bc{ア}のときに最小値\Bc{イ}をとる.\\\\
\tokeini 等式$
\begin{center}
    $\dfrac{3x^2-x+4}{\p{x+1}^3}=\dfrac{a}{\p{x+1}^3}+\dfrac{b}{\p{x+1}^2}+\dfrac{c}{x+1}$
\end{center}
が$xについての恒等式となるような定数a,b,cは,a=\Bc{ウ},b=\Bc{エ},c=\Bc{オ}である.\\\\
\tokeisan さいころを3回投げて出る目をすべてかけた数が4の倍数となる確率は\Bc{カ}である.\\\\
\tokeishi \theta=\dfrac{\pi}{12}のとき,\dfrac{1}{\tan\theta}-\tan\theta の値は\Bc{キ}である.\\\\
\tokeigo 初項と第2項がそれぞれa_1=1,a_2=1である数列\B{a_n}は,n\geq2のとき等式$
\begin{center}
    $a_{n+1}=a_1+a_2+\cdots+a_n$
\end{center}
をみたす.$n\geq3のとき,a_nをnを用いて表すとa_n=\Bc{ク}である.\\\\
\tokeiroku 0\leq x\leq1の範囲においてf(x)\geq0である2次関数f(x)=ax^2+bは,等式$
\begin{center}
    $f(x)\p{\dint{0}{1}f(t)dt}=x^2+5$
\end{center}
を満たす.$このとき,定数a,bは,a=\Bc{ケ},b=\Bc{コ}である.$


\newpage

    \section{2025年度}










\end{document}